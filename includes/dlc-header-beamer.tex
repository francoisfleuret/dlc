\usepackage[utf8]{inputenc}
\usepackage{overpic}
\usepackage{pgfplots}
\usepackage{xspace}
\usepackage{verbatimbox} % To insert verbatim inside \note

%%%%%%%%%%%%%%%%%%%%%%%%%%%%%%%%%%%%%%%%%%%%%%%%%%%%%%%%%%%%%%%%%%%%%%

%% \bibliographystyle{abbrvnat}
\bibliographystyle{includes/abbrvnatforbeamer}
\usepackage{multirow}

\usepackage{microtype}
\usepackage{amsmath}
\usepackage{amssymb}
\usepackage{dsfont}
\usepackage{verbatim}
\usepackage{upquote}
\usepackage{anyfontsize}
\usepackage[round]{natbib}
\usepackage{fancyvrb}
\usepackage{mathtools}
\usepackage{etex}
\usepackage{xparse}
\usepackage{xifthen}
%% \usepackage{BOONDOX-cal}

\def\theurl{\url{https://fleuret.org/dlc/}}
\def\theauthor{Fran\c cois Fleuret}
\def\coursetitle{Deep learning}

\makeatletter
\DeclareFontEncoding{LS1}{}{}
\DeclareFontSubstitution{LS1}{stix}{m}{n}
\DeclareMathAlphabet{\mathcal}{LS1}{stixscr}{m}{n}
\makeatother

\def\relu{\operatorname{ReLU}}
\def\softmax{\operatorname{softmax}}
\def\sigmoid{\operatorname{sigm}}
\def\sample{\operatorname{sample}}
\def\diag{\operatorname{diag}}
\def\sign{\operatorname{sign}}
\def\argmax{\operatornamewithlimits{argmax}}
\def\argmin{\operatornamewithlimits{argmin}}
\def\trace{\operatorname{Tr}}

% Olivier: I changed {\it #1} to \textit{#1} because there are
% different and does not mess up with serif/sansserif
%
% https://tex.stackexchange.com/questions/8053/is-there-a-difference-between-textit-and-itshape
\newcommand{\latin}[1]{\textit{#1}}
% \newcommand{\latin}[1]{{\it #1}}

\newcommand{\materialsurl}[1]{\url{https://fleuret.org/dlc/#1}}

\newcommand{\DATAVAR}{\mathbf{{\cal D}}}
\newcommand{\DATAVAL}{\mathbf{d}}
\newcommand{\BD}{\mathbf{D}}
\newcommand{\LL}{\mathcal{L}}
\newcommand{\Ll}{\mathcal{l}}
\newcommand{\RR}{\mathbb{R}}
\newcommand{\Lh}{\mathcal{h}}
\newcommand{\transpose}{^{\top}}

\newcommand{\dd}[2]{\frac{\partial {#1}}{\partial {#2}}}
%%\newcommand{\jacob}[2]{\left[ \dd{#1}{{#2}} \right]}
\newcommand{\gradient}[2]{\left[ \dd{#1}{{#2}} \right]}
%% \newcommand{\jacob}[2]{\mathbf{J}_{#1}}
\newcommand{\DD}[2]{\left[ \frac{\partial {#1}}{\partial {#2}} \right]}
\newcommand{\lbb}{\left[\!\!\left[}
\newcommand{\rbb}{\right]\!\!\right]}

\newcommand{\parname}[1]{{(#1)}}
\newcommand{\naming}[1]{{\text{(#1)}}}

%% \newcommand{\parname}[1]{{\vec{#1}}}
%% \newcommand{\parname}[1]{{\widetilde{#1}}}
%% \newcommand{\parname}[1]{{\tilde{#1}}}
%% \newcommand{\parname}[1]{{(#1)}}

%% \DeclareMathOperator*{\expect}{\mathds{E}}
%% \DeclareMathOperator*{\variance}{\mathds{V}}
%% \DeclareMathOperator*{\empexpect}{\hat{\mathds{E}}}
%% \DeclareMathOperator*{\mutinf}{\mathds{I}}
%% \DeclareMathOperator*{\empmutinf}{\hat{\mathds{I}}}
%% \DeclareMathOperator*{\entropy}{\mathds{H}}
%% \DeclareMathOperator*{\empentropy}{\hat{\mathds{H}}}

\def\given{\,\middle\vert\,}
\newcommand{\proba}{{P}}
\newcommand{\seq}{{S}}
%% \newcommand{\proba}{\mathds{P}}
\newcommand{\expect}{\mathds{E}}
\newcommand{\variance}{\mathds{V}}
\newcommand{\empexpect}{\hat{\mathds{E}}}
\newcommand{\mutinf}{\mathds{I}}
\newcommand{\empmutinf}{\hat{\mathds{I}}}
\newcommand{\entropy}{\mathds{H}}
\newcommand{\empentropy}{\hat{\mathds{H}}}
\newcommand{\ganG}{\mathbf{G}}
\newcommand{\ganD}{\mathbf{D}}
\newcommand{\ganF}{\mathbf{F}}

\newcommand{\dkl}{\mathds{D}_{\mathsf{KL}}}
\newcommand{\djs}{\mathds{D}_{\mathsf{JS}}}

\newcommand*{\vertbar}{\rule[-1ex]{0.5pt}{2.5ex}}
\newcommand*{\horzbar}{\rule[.5ex]{2.5ex}{0.5pt}}

\usepackage{xcolor}

%%%%%%%%%%%%%%%%%%%%%%%%%%%%%%%%%%%%%%%%%%%%%%%%%%%%%%%%%%%%%%%%%%%%%%
%% Colors

\definecolor{blue}{rgb}{0.0,0.0,0.55}
\definecolor{paleblue}{rgb}{0.50,0.50,1.00}
\definecolor{darkblue}{rgb}{0.10,0.10,0.70}

\definecolor{red}{rgb}{1.0,0.0,0.0}
\definecolor{palered}{rgb}{1.00,0.50,0.50}
\definecolor{darkred}{rgb}{0.75,0.0,0.0}

\definecolor{green}{rgb}{0.0,0.50,0.0}
\definecolor{palegreen}{rgb}{0.5,1.00,0.5}
\definecolor{darkgreen}{rgb}{0.0,0.5,0.0}

\definecolor{dimmed}{rgb}{0.8,0.8,0.8}

\definecolor{orange}{rgb}{1.0,0.6,0.0}
\definecolor{bluegray}{rgb}{0.1,0.2,0.7}

\newcommand{\blue}{\color{blue}}
\newcommand{\darkblue}{\color{darkblue}}
\newcommand{\paleblue}{\color{paleblue}}
\newcommand{\green}{\color{green}}
\newcommand{\darkgreen}{\color{darkgreen}}
\newcommand{\palegreen}{\color{palegreen}}
\newcommand{\red}{\color{red}}
\newcommand{\darkred}{\color{darkred}}
\newcommand{\palered}{\color{palered}}
\newcommand{\black}{\color{black}}

%%%%%%%%%%%%%%%%%%%%%%%%%%%%%%%%%%%%%%%%%%%%%%%%%%%%%%%%%%%%%%%%%%%%%%

\newcommand{\quotepaper}[2]{%
\begin{center}%
%
\begin{minipage}{0.95\textwidth}
``#2''
\end{minipage}%
%
\end{center}

\vspace*{-1ex}

\acksource{\citep{#1}}
}

%%%%%%%%%%%%%%%%%%%%%%%%%%%%%%%%%%%%%%%%%%%%%%%%%%%%%%%%%%%%%%%%%%%%%%

\newcommand{\quotesource}[2]{%
\begin{center}%
%
\begin{minipage}{0.95\textwidth}
``#2''
\end{minipage}%
%
\end{center}

\vspace*{-1ex}

\acksource{#1}
}

%%%%%%%%%%%%%%%%%%%%%%%%%%%%%%%%%%%%%%%%%%%%%%%%%%%%%%%%%%%%%%%%%%%%%%

\definecolor{codecolor}{rgb}{0.0,0.0,0.0}
\definecolor{codecolor2}{rgb}{0.0,0.0,0.0}

%% One keyword. The heavy magic to allow the percent symbol
%% \makeatletter
%% \newcommand{\kw}{\catcode`\%=12\@kw}
%% \DeclareRobustCommand{\@kw}[1]{{{\color{codecolor}\tt \detokenize{#1}}\catcode`\%=14}{}}
%% \makeatother
\DeclareRobustCommand{\kw}[1]{{\color{codecolor}\tt \detokenize{#1}}}

%% Inline piece of code

\DefineVerbatimEnvironment{rawsrc}{BVerbatim}{fontsize=\small,baselinestretch=1.0,formatcom=\color{codecolor}}

%% Sub-part of a file

\ifnum\pdfshellescape=1
\NewDocumentCommand{\rawsrcexcerpt}{mmmO{}O{1}}{%
  \BVerbatimInput[fontsize=\small,baselinestretch=1.0,formatcom=\color{codecolor},#4]{|"< #1 awk '/#2/{flag=1;next}/#3/{flag=0}flag {print substr($0, #5)}' | grep -v HIDE_IN_SLIDE"}%
}
\else
\NewDocumentCommand{\rawsrcexcerpt}{mmmO{}O{1}}{%
  {\color{darkred} \tt Source snippet omitted because shell escape is disabled.}%
}
\fi

%%%%%%%%%%%%%%%%%%%%%%%%%%%%%%%%%%%%%%%%%%%%%%%%%%%%%%%%%%%%%%%%%%%%%%

\newenvironment{warning}{%
\raisebox{-8pt}{%
\begin{tikzpicture}[scale=0.6]
\node at (0, 0) {\huge\textrm{\textbf{!}}};
\draw[rounded corners=1pt,line width=1.5pt,darkred] (-0.6, -0.47) -- ++(1.2, 0) -- ++(-0.6, 1.2) -- cycle;
\end{tikzpicture}%
}%
%% {\darkred \huge {\fontencoding{U}\fontfamily{futs}\selectfont\char 66\relax}}%
%
\hspace*{0.75em}%
%
\begin{minipage}{0.9\textwidth}
}{%
\end{minipage}%
}

%%%%%%%%%%%%%%%%%%%%%%%%%%%%%%%%%%%%%%%%%%%%%%%%%%%%%%%%%%%%%%%%%%%%%%

\newcommand{\fixedheightbox}[2]{%
   \sbox{0}{\parbox{\textwidth}{#2}}
    \ifdim\dimexpr\ht0+\dp0<#1
    \dp0\dimexpr#1-\ht0\fi
    \usebox{0}%
}

\makeatletter
\newcommand{\smallerthantiny}{\@setfontsize{\smallerthantiny}{4pt}{5pt}}
\makeatother

%%%%%%%%%%%%%%%%%%%%%%%%%%%%%%%%%%%%%%%%%%%%%%%%%%%%%%%%%%%%%%%%%%%%%%
%% The \draft command

\newcounter{nbdrafts}
\setcounter{nbdrafts}{0}
\makeatletter
\newcommand{\checknbdrafts}{
\ifnum \thenbdrafts > 0
\@latex@warning@no@line{**********************************************************************}
\@latex@warning@no@line{* The document contains \thenbdrafts \space draft note(s)}
\@latex@warning@no@line{**********************************************************************}
\fi}
\newcommand{\draft}[1]{\addtocounter{nbdrafts}{1}{\color{red} #1}}
\makeatother

\newcommand{\vstretch}{\vspace*{\stretch{1}}}
\newcommand{\hstretch}{\hspace*{\stretch{1}}}

%%%%%%%%%%%%%%%%%%%%%%%%%%%%%%%%%%%%%%%%%%%%%%%%%%%%%%%%%%%%%%%%%%%%%%
% Across files references, e.g. \dlclabel{autograd} in the slide file
% about autograd, and \dlcref{autograd} where we want to refer to it

\newcommand{\dlclabel}[1]{%
\newwrite\file
\immediate\openout\file=#1.dlcref
%% \immediate\write\file{\dlclecturenumber.\dlcdecknumber.\,``\detokenize\expandafter{\dlcdecktitle}''}
%\immediate\write\file{\dlclecturenumber.\dlcdecknumber.\detokenize\expandafter{\,``\dlcdecktitle''}}
\immediate\write\file{\dlclecturenumber.\dlcdecknumber. ``\dlcdecktitle''}
\immediate\closeout\file
}

\newcommand{\dlcref}[1]{%
\InputIfFileExists{#1.dlcref}{}{[dlcref #1 undefined]}\unskip
%\@input{#1.dlcref}\unskip
}%

%%%%%%%%%%%%%%%%%%%%%%%%%%%%%%%%%%%%%%%%%%%%%%%%%%%%%%%%%%%%%%%%%%%%%%

% To include a .pdf image which itself includes a raster image
% relative to itself
%
% https://tex.stackexchange.com/a/282110
\newcommand\inputpgf[2]{{
\let\pgfimageWithoutPath\pgfimage
\renewcommand{\pgfimage}[2][]{\pgfimageWithoutPath[##1]{#1/##2}}
\input{#1/#2}
}}


\usepackage{tikz}

\usetikzlibrary{positioning,fit,backgrounds}
\usetikzlibrary{arrows.meta,decorations.pathreplacing}
\usetikzlibrary{calc}
\usetikzlibrary{shapes,calc,intersections}
\usetikzlibrary{patterns}
%% \usetikzlibrary{shapes.multipart}

\usetikzlibrary{arrows}
%% \tikzset{>=angle 90}

\definecolor{nn-data}   {rgb}{0.90, 0.95, 1.00}
\definecolor{nn-param}  {rgb}{1.00, 0.90, 0.50}
\definecolor{nn-process}{rgb}{0.80, 1.00, 0.80}

\tikzset{
  pics/box/.style args={#1/#2/#3/#4/#5/#6}{
    code={
      \pgfmathsetmacro{\slant}{0.35}
      \pgfmathsetmacro{\width}{#1}
      \pgfmathsetmacro{\height}{#2}
      \pgfmathsetmacro{\thickness}{#3}
      \pgfmathsetmacro{\lwidth}{#4}
      \pgfmathsetmacro{\lheight}{#5}
      \pgfmathsetmacro{\lthickness}{#6}
      \pgfmathsetmacro{\labelgap}{0.15}

      \pgfmathsetmacro{\centerx}{0}
      \pgfmathsetmacro{\centery}{\height * 0.5 + \width * 0.5 * \slant}

      % Filled body

      \draw[fill] ( - \centerx, - \centery)
      -- ++(0.0, \height)
      -- ++(\slant * 0.5 * \width, 0.5 * \width)
      -- ++(\thickness, 0.0)
      -- ++(0, -\height)
      -- ++(- \slant * 0.5 * \width, -0.5 * \width)
      -- ++(-\thickness, 0)
      ;

      % Additional edges

      \draw  ( - \centerx, - \centery) ++(0.0, \height)
      -- ++(\thickness, 0.0) -- ++(\slant * 0.5 * \width, 0.5 * \width)
      ;

      \draw  ( - \centerx, - \centery) ++(\thickness, \height)
      -- ++(0.0, -\height)
      ;

      % Axis length labels

      \ifthenelse
      {\equal{\lwidth}{}}{}
      {
        \draw[<->]  ( - \centerx, - \centery) ++(0.0, \height) ++(-\labelgap * .7071, \labelgap * .7071)
        -- ++(\slant * 0.5 * \width, 0.5 * \width) node[midway, above left] {\scriptsize \lwidth};
      }

      \ifthenelse
      {\equal{\lheight}{}}{}
      {
        \draw[<->]  ( - \centerx, - \centery) ++(-\labelgap, 0.0)
        -- ++(0.0, \height) node[midway, left] {\scriptsize \lheight};
      }

      \ifthenelse
      {\equal{\lthickness}{}}{}
      {
        \draw[<->]  ( - \centerx, - \centery) ++(0.0, -\labelgap)
        -- ++(\thickness, 0.0) node[midway, below] {\scriptsize \lthickness};
      }

      % Anchor points

      \coordinate (-center) at (\thickness + \slant * 0.25 * \width, 0.0);

      \coordinate (-follow-tight) at (\thickness + 0.2, 0.0);

      \coordinate (-follow-close) at (\thickness + \slant * 0.5 * \width, 0.0);

      \coordinate (-follow) at (\thickness + \slant * 0.2 * \width + 1.0, 0.0);

      \coordinate (-above-back)  at (-\centerx + \slant * 0.5 * \width + \thickness * 0.5, \height - \centery + 0.5 * \width + 0.3);

      \coordinate (-above)  at (-\centerx + \thickness * 0.5, \height - \centery + 0.5 * \width + 0.3);

      \coordinate (-below)  at (-\centerx + \thickness * 0.5, -\centery - 0.5);
    }
  },
  pics/box/.default=0.5/1/1/1/1/1
}

%%%%%%%%%%%%%%%%%%%%%%%%%%%%%%%%%%%%%%%%%%%%%%%%%%%%%%%%%%%%%%%%%%%%%%

%% \newcommand{\intint}[1]{[\![#1]\!]}
\newcommand{\intint}[1]{[\![#1]\!]}

\newcommand{\cube}[6]{
    \draw[#1,#2] #3 -- ++#4 -- ++#5 -- ++#6 -- ++($(0, 0) - #4$) -- ++($(0, 0) - #5$) -- ++($(0, 0) - #6$);
    \draw[#1] #3 ++#4 -- ++#6 -- ++#5;
    \draw[#1] #3 ++#6 -- ++#4;
}

%%%%%%%%%%%%%%%%%%%%%%%%%%%%%%%%%%%%%%%%%%%%%%%%%%%%%%%%%%%%%%%%%%%%%%

\newcommand{\oneconv}[3]{
  \uncover<#1>{
    \cube{draw=black,thick}{fill=black!15}{#3}{(0, 1)}{(0.4, 0.8)}{(1, 0)}
  }

  \uncover<#1-#2>{
    \cube{draw=green,thick}{fill=white}{#3 ++(7.4, 0.4)}{(0, 0.6)}{(0.2, 0.4)}{(0.33333, 0)}
  }
}

\newcommand{\onepool}[3]{
  \uncover<#1>{
    \cube{draw=black,thick}{fill=black!15}{#3}{(0, 1)}{(0.4, 0.8)}{(1, 0)}
  }

  \uncover<#1-#2>{
    \cube{draw=green,thick}{fill=white}{#3 ++(7.4, 0.4)}{(0, 0.6)}{(0.2, 0.4)}{(0.33333, 0)}
  }
}

%%%%%%%%%%%%%%%%%%%%%%%%%%%%%%%%%%%%%%%%%%%%%%%%%%%%%%%%%%%%%%%%%%%%%%

\newcommand{\drawvector}[1]{
%% \raisebox{0.75cm}{\Large $\Bigg($}
\begin{tikzpicture}[scale=0.2]
  \draw[draw=none] (0, -4) -- (0, 4);
  \edef\xdraw{0}
  \draw[black!20,thin] (0, -0.2) -- ++(0, 0.4);
  \foreach \y in { #1 }{
    \pgfmathparse{\xdraw+1}
    \xdef\xdraw{\pgfmathresult}
    \draw[black!20,thin] (\xdraw, -0.2) -- ++(0, 0.4);
  }
  \draw[black!20,thin] (-0.1, 0) -- (\xdraw, 0) ++(0.1, 0);
  \edef\xdraw{0}
  \foreach \y in { #1 }{
    \draw[] (\xdraw, 0) -- ++(0, \y) -- ++(1.0, 0.0) -- ++(0, -\y);
    %% \draw[] (\xdraw, \y) -- ++(0.05, 0) -- ++(0.9, 0.0);
    %% \draw[thick] (\xdraw,\y) +(0.05, 0) -- ++(0.9, 0.0);
    \pgfmathparse{\xdraw+1}
    \xdef\xdraw{\pgfmathresult}
  }
\end{tikzpicture}%
%% \raisebox{0.75cm}{\Large $\Bigg)$}
}

%%%%%%%%%%%%%%%%%%%%%%%%%%%%%%%%%%%%%%%%%%%%%%%%%%%%%%%%%%%%%%%%%%%%%%

%%%%%%%%%%%%%%%%%%%%%%%%%%%%%%%%%%%%%%%%%%%%%%%%%%%%%%%%%%%%%%%%%%%%%%
% A command to illustrate the convolutional layer output size

\newcommand{\convscheme}[4]{
\begin{tikzpicture}[scale=0.15]
\draw (0, 0) -- (#1, 0);
\draw[<->] (0, 1.75) -- ++(#1, 0) node[midway,above] {$#1$};

\draw[fill=green!25] (#1, 0.00) ++(-#2, 0.0) rectangle ++(#2, 1.0);

\foreach \x in { 1,...,#1 }
  \draw[thin] (\x, 0.0) ++(-1, 0) rectangle ++(1, 1);

\draw[] (0, 0.00) ++(0, -0.5) -- ++({#3*(#4-1)+1}, 0.0) node[midway,below] {$\times #4$};

\foreach \x in { 1,...,#4 }
  \draw[fill=black] ({#3*(\x-1)}, 0.5) ++(0.5, 0) circle(4pt);
\end{tikzpicture}
}

\newcommand{\convtransposescheme}[4]{
\begin{tikzpicture}[scale=0.15]
\draw (0, 0) -- (#1, 0);

\draw[fill=green!25] (#1, 0.00) ++(-#2, 0.0) rectangle ++(#2, 1.0);

\foreach \x in { 1,...,#1 }
  \draw[thin] (\x, 0.0) ++(-1, 0) rectangle ++(1, 1);

\draw[] (0, 0.00) ++(0, 1.5) -- ++({#3*(#4-1)+1}, 0.0) node[midway,above] {$\times #4$};
%% \draw[<->] (#1, 0.00) ++(-#2, -0.75) -- ++(#2, 0.0) node[midway,below] {$#2$};
\draw[<->] (0, -0.75) -- ++(#1, 0.00) node[midway,below] {$#1$};

\foreach \x in { 1,...,#4 }
  \draw[fill=black] ({#3*(\x-1)}, 0.5) ++(0.5, 0) circle(4pt);
\end{tikzpicture}
}

%%%%%%%%%%%%%%%%%%%%%%%%%%%%%%%%%%%%%%%%%%%%%%%%%%%%%%%%%%%%%%%%%%%%%%

\tikzset{>={Straight Barb[angle'=80,scale=1.1]}}

\tikzset{
  value/.style    ={ font=\scriptsize, rectangle, draw=black!50, fill=white,   thick,
                     inner sep=3pt, inner xsep=2pt, minimum size=10pt, minimum height=20pt },
  parameter/.style={ font=\scriptsize, rectangle, draw=black!50, fill=blue!15, thick,
                     inner sep=0pt, inner xsep=2pt, minimum size=10pt, minimum height=20pt },
  operation/.style={ font=\scriptsize, rectangle,    draw=black!50, fill=green!30, thick,
                     inner sep=3pt, minimum size=10pt, minimum height=20pt },
  flow/.style={->,shorten <= 1pt,shorten >= 1pt, draw=black!50, thick},
%
  f2f/.style={draw=black!50, thick},
  v2f/.style={{Bar[width=1.5mm]}-,shorten <= 0.75pt,draw=black!50, thick},
  f2v/.style={->,shorten >= 0.75pt,draw=black!50, thick},
  v2v/.style={{Bar[width=1.5mm]}->,shorten <= 0.75pt,shorten >= 0.5pt,draw=black!50, thick},
%
%
  df2f/.style={draw=black, thick},
  dv2f/.style={{Bar[width=1.5mm]}-,shorten <= 0.75pt,draw=black, thick},
  df2v/.style={->,shorten >= 0.75pt,draw=black, thick},
  dv2v/.style={{Bar[width=1.5mm]}->,shorten <= 0.75pt,shorten >= 0.5pt,draw=black, thick},
%
  differential/.style    ={ font=\small, rectangle, draw=black!50,               thick,
                     inner sep=3pt, inner xsep=2pt, minimum size=10pt, minimum height=20pt, fill=yellow!80 },
  dflow/.style={->,shorten <= 1pt,shorten >= 1pt, draw=black, thick}
}

\newcommand{\nophone}{
\begin{tikzpicture}

\draw[fill=black] (0, 0) to (0, 7) to [out=6,in=174] (4, 7) to (4, 0) to [out=186,in=354] (0, 0);

\draw[fill=white] (0.2, 0.75) rectangle (3.8, 6.35);
\draw[fill=white] (2, 0.27) circle (0.2);
\draw[fill=white] (1, 6.7) circle (0.1);

\draw[line width=33pt,color=white] (2, 3.5) circle (5cm);
\draw[line width=33pt,color=white] (2, 3.5) ++(-3.3, -3.3) -- ++(6.6, 6.6);
\draw[line width=23pt,color=red] (2, 3.5) circle (5cm);
\draw[line width=23pt,color=red] (2, 3.5) ++(-3.3, -3.3) -- ++(6.6, 6.6);

\end{tikzpicture}
}

%\input{includes/source-highlight}

\pgfplotsset{compat=1.11}

%%%%%%%%%%%%%%%%%%%%%%%%%%%%%%%%%%%%%%%%%%%%%%%%%%%%%%%%%%%%%%%%%%%%%%
%% The witharrows mess due to my old Debian

\usepackage{witharrows} % To make explanatory arrows between equation lines

%% \usepackage{environ}
%% \NewEnviron{DispWithArrows*}{%
%% \begin{align}
%% \BODY
%% \end{align}
%% }
%% \newcommand{\Arrow}[1]{}

%%%%%%%%%%%%%%%%%%%%%%%%%%%%%%%%%%%%%%%%%%%%%%%%%%%%%%%%%%%%%%%%%%%%%%
%% Do not skip frame numbers
\usepackage{etoolbox}
\makeatletter
\pretocmd{\beamer@@@@frame}{\alt<#1>{}{\beamer@noframenumberingtrue}}{}{}
\makeatother
%%%%%%%%%%%%%%%%%%%%%%%%%%%%%%%%%%%%%%%%%%%%%%%%%%%%%%%%%%%%%%%%%%%%%%

%%%%%%%%%%%%%%%%%%%%%%%%%%%%%%%%%%%%%%%%%%%%%%%%%%%%%%%%%%%%%%%%%%%%%%
% To make an index
\newenvironment{theindex}{}{}
\usepackage{imakeidx}
\usepackage{fp}

\makeatletter
% Global redefinition of indexentry to use section, then page
\newcounter{tmpcounter}
\renewcommand{\imki@wrindexentrysplit}[3]{%
  %%%%%%%%%%%%%%%%%%%%%%%%%%%%%%%%%%%%%%%%%%%%%%%%%%%
  % To make the items easily sortable (integers) without redefining too
  % much xindy and co. an index is referenced with:
  % 13002007 : lecture 13, course 2, frame 7
  % 1000000 * lecture + 1000 * deck + frame
  %%%%%%%%%%%%%%%%%%%%%%%%%%%%%%%%%%%%%%%%%%%%%%%%%%%
  % End all following lines with '%' otherwise,
  % an extra space is generated by \index
  %%%%%%%%%%%%%%%%%%%%%%%%%%%%%%%%%%%%%%%%%%%%%%%%%%%
  \setcounter{tmpcounter}{\dlclecturenumber}%
  \ifnum \thetmpcounter > 999%
  \PackageError{dlc}{Lecture number (\dlclecturenumber) should be < 1000 for index encoding}{}%
  \fi%
  \setcounter{tmpcounter}{\dlcdecknumber}%
  \ifnum \thetmpcounter > 999%
  \PackageError{dlc}{Deck number (\dlcdecknumber) should be < 1000 for index encoding}{}%
  \fi%
  \FPeval{\indexid}{clip(1000000*\dlclecturenumber+1000*\dlcdecknumber+\insertframenumber)}%
  % \expandafter\protected@write\csname#1@idxfile\endcsname{}{\string\indexentry{#2}{\lecturenumber.\coursenumber:\insertframenumber}}%
  \expandafter\protected@write\csname#1@idxfile\endcsname{}{\string\indexentry{#2}{\indexid}}%
}%
\makeatother
%%%%%%%%%%%%%%%%%%%%%%%%%%%%%%%%%%%%%%%%%%%%%%%%%%%%%%%%%%%%%%%%%%%%%%

\makeindex

\let\oldindex\index % save old definition to prevent recursion
\renewcommand*\index[1]{\oldindex{#1|formatsliderange}}
% \renewcommand*\index[1]{\oldindex{#1|formatsliderange}\ignorespaces}
%%%%%%%%%%%%%%%%%%%%%%%%%%%%%%%%%%%%%%%%%%%%%%%%%%%%%%%%%%%%%%%%%%%%%%


% %%~~~~~~~~~~~~~~~~~~~~~~~~~~~~~~~~~~~~~~~~~~~~~~~~~~~~
% \mode<handout>{
%   \usepackage{pgfpages}
%   %% \pgfpagesuselayout{16 on 1}[a4paper,border shrink=0mm,landscape]
%   %% \pgfpageslogicalpageoptions{1}{border code=\pgfusepath{stroke}}
%   %% \pgfpageslogicalpageoptions{2}{border code=\pgfusepath{stroke}}
%   %% \pgfpageslogicalpageoptions{3}{border code=\pgfusepath{stroke}}
%   %% \pgfpageslogicalpageoptions{4}{border code=\pgfusepath{stroke}}
%   \pgfpagesuselayout{2 on 1}[a4paper,border shrink=4mm]
% }
% %% \setbeamercovered{transparent}
% %%~~~~~~~~~~~~~~~~~~~~~~~~~~~~~~~~~~~~~~~~~~~~~~~~~~~~

%%~~~~~~~~~~~~~~~~~~~~~~~~~~~~~~~~~~~~~~~~~~~~~~~~~~~
\usepackage{makecell}
%% To get handout on top of A4 page and not below, and to handle the
%% fact that some slides don't have notes.

\usepackage{ragged2e}  % justify text

\mode<handout>{
  % \setbeamercolor{background canvas}{bg=gray!10} % Olivier: debug handout
  \usepackage{pgfpages}
  %% \pgfpagesuselayout{16 on 1}[a4paper,border shrink=0mm,landscape]
  %% \pgfpageslogicalpageoptions{1}{border code=\pgfusepath{stroke}}
  %% \pgfpageslogicalpageoptions{2}{border code=\pgfusepath{stroke}}
  %% \pgfpageslogicalpageoptions{3}{border code=\pgfusepath{stroke}}
  %% \pgfpageslogicalpageoptions{4}{border code=\pgfusepath{stroke}}
  \pgfpagesuselayout{2 on 1}[a4paper,border shrink=3mm]

  \ifdefined\withoutnotes
    \ifdefined\withnotes
      \setbeameroption{show notes}
    \fi
  \else
    \setbeameroption{show notes}
    %% \pgfpageslogicalpageoptions{1}{border code=\pgfusepath{stroke}}
  \fi

  \setbeamertemplate{note page}{%

    \insertnote
  }
}
%%~~~~~~~~~~~~~~~~~~~~~~~~~~~~~~~~~~~~~~~~~~~~~~~~~~~

\catcode`\%=12
\newcommand\percentsymbol{%}
\catcode`\%=14

\def\closingmessage{The End}

%% \def\theauthor{Undefined author} % Author
%% \def\theurl{https://fleuret.org/dlc/}
%% \def\coursetitle{Undefined title}

\def\dlcdecktitle{Undefined subtitle}
%\def\coursenumber{}
\def\dlclecturenumber{} % for index
\def\dlcdecknumber{} % for index \lecturenb.\coursenb

%% \def\draftwarning{{\small \bf Draft, do not distribute}}
\def\draftwarning{}
%% \def\thedate{\scriptsize [version of: \today]}
\ifnum\pdfshellescape=1
%% \def\thedate{\small \input{|"date --utc -r \jobname.tex"}}
\def\thedate{\small \input{|"date +'\percentsymbol b \percentsymbol e, \percentsymbol Y' -r \jobname.tex"}}
\else
\def\thedate{\small Shell escape disabled}
\fi

\setlength{\parindent}{0cm}
\setlength{\parskip}{3ex}

\newcommand{\MM}{\mathcal M}
\newcommand{\TT}{\mathcal T}
%% \newcommand{\MM}{\mathscr M}
%% \newcommand{\TT}{\mathscr T}

\newcommand{\RD}{\mathbb{R}^D}
\newcommand{\st}{\,\text{s.t.}\,}

%% \usepackage{ebgaramond-maths}
%% \usepackage[T1]{fontenc}

%% \usepackage[round]{natbib}
%% \usepackage{natbib}
%% \bibliographystyle{abbrvnat}
%% \setcitestyle{authoryear,open={((},close={))}}

\usepackage{calc}

%% \usepackage{biblatex}
%% \AtEveryBibitem{%
  %% \clearfield{url}%
%% }

\newcommand{\mathand}{ \ \text{and} \ }

%% \tikzset{
  %% value/.style    ={ font=\small, rectangle, draw=black!50,               thick,
                     %% inner sep=0pt, inner xsep=2pt, minimum size=10pt },
  %% parameter/.style={ font=\small, rectangle, draw=black!50, fill=blue!30, thick,
                     %% inner sep=0pt, inner xsep=2pt, minimum size=10pt },
  %% operation/.style={ font=\scriptsize, circle,    draw=black!50, fill=blue!30, thick,
                     %% inner sep=0pt, minimum size=10pt },
  %% flow/.style={->,shorten <= 1pt,shorten >= 1pt, draw=black!50, thick}
%% }

%%%%%%%%%%%%%%%%%%%%%%%%%%%%%%%%%%%%%%%%%%%%%%%%%%%%%%%%%%%%%%%%%%%%%%
%% Font Open Sans light

%% \usepackage[default]{opensans}
%% \usepackage{cmbright}

%% \renewcommand{\familydefault}{fos}
%% \renewcommand{\seriesdefault}{l}
%% \renewcommand{\bfdefault}{sb}

%%%%%%%%%%%%%%%%%%%%%%%%%%%%%%%%%%%%%%%%%%%%%%%%%%%%%%%%%%%%%%%%%%%%%%
%% Colors

%% \mode<beamer>{
%% \setbeamercolor{math text}{fg=green}
\setbeamercolor{math text}{fg=bluegray}
\setbeamercolor{local structure}{fg=blue}
\definecolor{codecolor}{rgb}{0.0,0.4,0.5}%
\definecolor{codecolor2}{rgb}{0.0,0.7,0.0}%
%% }

%% \mode<handout>{
%% \setbeamercolor{math text}{fg=black}
%% \setbeamercolor{local structure}{fg=black}
%% }

%%%%%%%%%%%%%%%%%%%%%%%%%%%%%%%%%%%%%%%%%%%%%%%%%%%%%%%%%%%%%%%%%%%%%%
%% Remove the navigation bar, define the frame titles and footline

\newcommand{\titleemph}{}

\makeatletter
\define@key{beamerframe}{c}[true]{% centered
  \beamer@frametopskip=1em plus 1fill\relax%
  %% \beamer@frametopskip=0pt plus 1fill\relax%
  \beamer@framebottomskip=0pt plus 1fill\relax%
  \beamer@frametopskipautobreak=0pt plus .4\paperheight\relax%
  \beamer@framebottomskipautobreak=0pt plus .6\paperheight\relax%
  \def\beamer@initfirstlineunskip{}%
}
\makeatother

\setbeamertemplate{navigation symbols}{}
%% \setbeamersize{text margin left=2em,text margin right=2em}
\setbeamersize{text margin left=5em,text margin right=5em}
\setbeamertemplate{itemize item}{\raisebox{0.15ex}{\tiny \ensuremath{\black \bullet}}}
\setbeamertemplate{itemize subitem}{\raisebox{0.15ex}{\tiny \ensuremath{\black \bullet}}}
\setbeamerfont{frametitle}{size=\normalsize}

\setbeamercolor{normal text in math text}{fg=black}

%% \setlength{\textheight}{10cm}

\ifdefined\overrideframetitle\setbeamertemplate{frametitle}{\overrideframetitle}%
\else
\setbeamertemplate{frametitle}{
  \hskip-1.5em\usebeamerfont{frametitle}\titleemph{\insertframetitle}
  \vskip-0.2em
  \hskip-1.45em\usebeamerfont{framesubtitle}\titleemph{\insertframesubtitle}
}
\fi

\ifdefined\overridefootline\setbeamertemplate{footline}{\overridefootline}%
\else
\setbeamertemplate{footline}{\sffamily
  \vspace*{-1cm}

  \hspace{0.25em}
  %
  %ff% \makebox[20em][l]{\theauthor}
  \theauthor
  %
  \hstretch
  %
  \coursetitle{} / \dlclecturenumber.\dlcdecknumber.{} \dlcdecktitle{}
  %
  \hstretch
  %
  \insertframenumber \ / \inserttotalframenumber
  %ff% \makebox[20em][r]{\insertframenumber \ / \inserttotalframenumber}
  \hspace{0.25em}
  \vspace*{0.5ex}
}
\fi

%%%%%%%%%%%%%%%%%%%%%%%%%%%%%%%%%%%%%%%%%%%%%%%%%%%%%%%%%%%%%%%%%%%%%%

\newcommand{\sectiontitleframe}[1]{

\section{#1}

\begin{frame}

  \begin{center}

    %% \vspace*{\stretch{3}}

    \titleemph{\Large \color{blue} #1}

    %% \vspace*{\stretch{2}}

  \end{center}

\end{frame}
}

%%%%%%%%%%%%%%%%%%%%%%%%%%%%%%%%%%%%%%%%%%%%%%%%%%%%%%%%%%%%%%%%%%%%%%

\newcommand{\openingframe}{
\ifdefined\noopeningframe\else%
\setcounter{framenumberbackup}{\value{framenumber}}
\begin{frame}[plain]

  \center

  \vspace*{\stretch{10}}

  {\Large \color{blue} \coursetitle}

  {\Large \renewcommand{\baselinestretch}{1.5} \color{blue} \dlclecturenumber.\dlcdecknumber.\xspace \dlcdecktitle{}}
  %% \begin{minipage}{0.65\textwidth}\hyphenchar\font=-1
    %% \Large \renewcommand{\baselinestretch}{1.5} \color{blue} \dlcdecktitle
  %% \end{minipage}

  \vspace*{\stretch{4}}

  {\small \theauthor}\\[0.25em]

  {\small \theurl}\\[0.25em]

  %% {\small \thedate}

  \vspace*{\stretch{1}}

  %% \resizebox{1.0cm}{!}{\nophone}

  \draftwarning

  \vspace*{\stretch{1}}

  %%%%%%%%%%%%%%%%%%%%%%%%%%%%%%%%%%%%%%%%%%%%%%%%%%%%%%%%%%%%%%%%%%%%%%
  %
  \makebox[\textwidth][c]{
    %
    %% \raisebox{3pt}{\includegraphics[height=0.7cm]{logos/logo_idiap_flat.pdf}}
    %% %
    %% \hspace*{1.6cm}
    %
    \includegraphics[height=0.8cm]{logos/logo_unige.pdf}\ 
    %
    %% \hspace*{2.2cm}
    %% %
    %% \raisebox{6pt}{\includegraphics[height=0.525cm]{logos/logo_epfl_tex.pdf}}
    %% %
    %% \hspace*{0.3cm}
  }
  %
  %%%%%%%%%%%%%%%%%%%%%%%%%%%%%%%%%%%%%%%%%%%%%%%%%%%%%%%%%%%%%%%%%%%%%%

  %% \vspace*{5pt}

\end{frame}
\setcounter{framenumber}{\value{framenumberbackup}}
\fi
}

%%%%%%%%%%%%%%%%%%%%%%%%%%%%%%%%%%%%%%%%%%%%%%%%%%%%%%%%%%%%%%%%%%%%%%

\newcommand{\titleframe}[1]{
\begin{frame}[plain]

  \center

  \color{blue}

  #1

\end{frame}
}

%%%%%%%%%%%%%%%%%%%%%%%%%%%%%%%%%%%%%%%%%%%%%%%%%%%%%%%%%%%%%%%%%%%%%%

\newcounter{framenumberbackup}

\newcommand{\closingframe}{

\setcounter{framenumberbackup}{\value{framenumber}}

% Insert a final black slide
\mode<beamer>{
\setbeamertemplate{footline}{}

%% \addtocounter{framenumber}{-1}
{
\setbeamercolor{background canvas}{bg=black}
\begin{frame}[plain]{}
\vspace*{\stretch{6}}

\hstretch
%% {\color{darkgray} The end}
{\color{white} \closingmessage}
\hstretch

\vspace*{\stretch{5}}
\end{frame}
}
}

\setcounter{framenumber}{\value{framenumberbackup}}
}

%%%%%%%%%%%%%%%%%%%%%%%%%%%%%%%%%%%%%%%%%%%%%%%%%%%%%%%%%%%%%%%%%%%%%%

\newcommand{\bibliographyframe}{

\setcounter{framenumberbackup}{\value{framenumber}}

\addtolength{\headsep}{0.6cm}
\setlength{\textheight}{8cm}

\setbeamertemplate{footline}{}
\begin{frame}[allowframebreaks] %% frame 19 / 19

\vspace*{1ex}

\addtocounter{framenumber}{-1}

{\bf References}

\small

\bibliography{dlc}

\end{frame}

\setcounter{framenumber}{\value{framenumberbackup}}
}

%%%%%%%%%%%%%%%%%%%%%%%%%%%%%%%%%%%%%%%%%%%%%%%%%%%%%%%%%%%%%%%%%%%%%%

\usepackage{pifont}
\usepackage{includes/pdfpcnotes}

%%%%%%%%%%%%%%%%%%%%%%%%%%%%%%%%%%%%%%%%%%%%%%%%%%%%%%%%%%%%%%%%%%%%%%
% \note only used int handout mode
\let\oldnote\note
\usepackage{multicol}
% \renewcommand{\note}[1]{\mode<handout>{\oldnote{#1}}}

\setlength{\multicolsep}{0pt}
% \setlength{\columnsep}{3em}

\renewcommand{\note}[2][1]{\mode<handout>{\oldnote{%

      %% \vspace*{-5ex}

      \hrulefill

      \vspace*{1ex}

      \textbf{Notes}

      \ifnum#1=0%
%        \vspace*{-4ex}
        \twocolumn
        #2 \vfill
        \onecolumn
      \else
      \vspace*{1.1em} % To get same space as with \twocolumn
      \ifnum#1=1%
         #2
      \else
         \begin{multicols}{2}
         #2
         \end{multicols}
      \fi
   \fi
}}}

% \pnote only used int beamer mode
\let\oldpnote\pnote
\renewcommand{\pnote}[1]{\mode<beamer>{\oldpnote{#1}}}

% To make an empty note page if no \note is present (otherwise, this
% will shift the slide and notes: instead of having a slide[n] above
% notes[n], we end up with note[n-1] above slide[n])
%
% https://tex.stackexchange.com/questions/11708/run-macro-on-each-frame-in-beamer/11724#11724
\makeatletter
\def\beamer@framenotesbegin{% at beginning of slide
  \gdef\beamer@noteitems{}%
  \gdef\beamer@notes{{}}% used to be totally empty.
}
\makeatother
%%%%%%%%%%%%%%%%%%%%%%%%%%%%%%%%%%%%%%%%%%%%%%%%%%%%%%%%%%%%%%%%%%%%%%



\newcommand{\acksource}[1]{\hspace*{\stretch{1}}{#1}}

\setlength{\fboxsep}{0pt}
%% \setlength{\tabcolsep}{0.5mm}

%%%%%%%%%%%%%%%%%%%%%%%%%%%%%%%%%%%%%%%%%%%%%%%%%%%%%%%%%%%%%%%%%%%%%%
% the \cell command

\mode<handout>{
\newcommand{\cell}[2]{
  \hspace*{\stretch{1}}
  \hspace*{-1em}
  \begin{minipage}{#1\textwidth}
    \center
    #2
  \end{minipage}
  \hspace*{-1em}
  \hspace*{\stretch{1}}
}}

\mode<beamer>{\newcommand{\cell}[2]{\only<+>{\begin{center}#2\end{center}}}}

%%%%%%%%%%%%%%%%%%%%%%%%%%%%%%%%%%%%%%%%%%%%%%%%%%%%%%%%%%%%%%%%%%%%%%

\newcommand{\kwf}{\tt\scriptsize}

%%%%%%%%%%%%%%%%%%%%%%%%%%%%%%%%%%%%%%%%%%%%%%%%%%%%%%%%%%%%%%%%%%%%%%

\setbeamercolor{important}{bg=yellow}

\newenvironment{important}{%

\vspace*{1ex}

\setlength{\parskip}{0pt}%
\begin{beamercolorbox}[sep=0.5em]{important}
}
{
\end{beamercolorbox}
}

%%%%%%%%%%%%%%%%%%%%%%%%%%%%%%%%%%%%%%%%%%%%%%%%%%%%%%%%%%%%%%%%%%%%%%

\mode<beamer>{
\newcommand{\twographs}[2]{
\begin{frame}{}{}

\begin{center}
\includegraphics[trim=50pt 0 45pt 0]{materials0/#1}
\end{center}

\end{frame}

\begin{frame}{}{}

\begin{center}
\includegraphics[trim=50pt 0 45pt 0]{materials0/#2}
\end{center}

\end{frame}
}
}

\mode<handout>{
\newcommand{\twographs}[2]{
\begin{frame}{}{}

\makebox[\textwidth][c]{
\includegraphics[scale=0.6,trim=50pt 0 45pt 0]{materials0/#1}
\hspace*{1em}
\includegraphics[scale=0.6,trim=50pt 0 45pt 0]{materials0/#2}
}

\end{frame}
}
}


\mode<beamer>{
\newcommand{\twographstwo}[4]{
\begin{frame}{}{}

\begin{center}
\includegraphics[width=\textwidth]{materials/#1}
\end{center}

\end{frame}

\begin{frame}{}{}

\begin{center}
\includegraphics[width=\textwidth]{materials/#2}
\end{center}

\end{frame}
}
}

\mode<handout>{
\newcommand{\twographstwo}[4]{
\begin{frame}{}{}

\makebox[\textwidth][c]{
\includegraphics[width=0.5\textwidth]{materials/#1}
\hspace*{1em}
\includegraphics[width=0.5\textwidth]{materials/#2}
}

\note[#3]{#4}

\end{frame}
}
}

%%%%%%%%%%%%%%%%%%%%%%%%%%%%%%%%%%%%%%%%%%%%%%%%%%%%%%%%%%%%%%%%%%%%%%

\newcommand{\mnistblock}[2]{
\vspace*{-0.5em}

#1

\vspace*{-0.75em}

\includegraphics[scale=0.65]{#2}
}

%%%%%%%%%%%%%%%%%%%%%%%%%%%%%%%%%%%%%%%%%%%%%%%%%%%%%%%%%%%%%%%%%%%%%%

%%%%%%%%%%%%%%%%%%%%%%%%%
\usepackage{ulem}

\input ulem.sty\relax
\catcode`\@=11 %

\font\uwavefontold=lasy6
\font\uwavefont=lasyb10 scaled 652

\def\uwave{%
  \bgroup
    \markoverwith{%
      \lower4.5\p@\hbox{\color{darkred}\uwavefont\char58}%
    }%
  \ULon
}

%%%%%%%%%%%%%%%%%%%%%%%%%%%%%%%%%%%%%%%%%%%%%%%%%%%%%%%%%%%%%%%%%%%%%%
%%%% The attempt to make sansserif slides and garamond notes
%%%%%%%%%%%%%%%%%%%%%%%%%%%%%%%%%%%%%%%%%%%%%%%%%%%%%%%%%%%%%%%%%%%%%%

% \usepackage{sansmath}
% \sansmath
% \usepackage[bb=ams,cal=boondoxo,scr=zapfc]{mathalfa}
% \usefonttheme{serif}

% \BeforeBeginEnvironment{frame}{\sffamily}

% % To have Palatino/Garamond serif in math mode although the global
% % style of the slides are sans serif
% %
% % https://tex.stackexchange.com/questions/385013/how-to-set-temporary-math-font/385068#385068

% \DeclareMathVersion{varnormal}
% \DeclareMathVersion{varbold}
% \newcommand\txmath{\mathversion{normal}\unsansmath}
% \newcommand\txboldmath{\mathversion{bold}}
% \newcommand{\mdmath}{\mathversion{varnormal}\unsansmath}
% \newcommand\mdboldmath{\mathversion{varbold}}

% \renewcommand{\rmdefault}{ppl}

% %%% Math symbol fonts
% %%% some examples only
% % Math letters from txfonts and mdugm
% \SetSymbolFont{letters}{normal}{OML}{txmi}{m}{it}
% \SetSymbolFont{letters}{bold}{OML}{txmi}{bx}{it}
% \SetSymbolFont{letters}{varnormal}{OML}{ppl}{m}{it}
% \SetSymbolFont{letters}{varbold}{OML}{ppl}{b}{it}
% % Math operators
% % \SetSymbolFont{operators}{normal}{OT1}{ppl}{m}{n}
% \SetSymbolFont{operators}{normal}{OT1}{txr}{m}{n}
% \SetSymbolFont{operators}{bold}{OT1}{txr}{bx}{n}
% \SetSymbolFont{operators}{varnormal}{OT1}{ppl}{m}{n}
% \SetSymbolFont{operators}{varbold}{OT1}{ppl}{b}{n}
% % Math symbols
% \SetSymbolFont{symbols}{normal}{OMS}{txsy}{m}{n}
% % \SetSymbolFont{symbols}{bold}{OMS}{txsy}{bx}{n}
% \SetSymbolFont{symbols}{varnormal}{OMS}{ppl}{m}{n}
% % \SetSymbolFont{symbols}{varbold}{OMS}{ppl}{b}{n}
% % % Large symbols
% \SetSymbolFont{largesymbols}{normal}{OMX}{txex}{m}{n}
% % \SetSymbolFont{largesymbols}{bold}{OMX}{txex}{bx}{n}
% \SetSymbolFont{largesymbols}{varnormal}{OMX}{ppl}{m}{n}
% % \SetSymbolFont{largesymbols}{varbold}{OMX}{ppl}{b}{n}

% %%% Math alphabets, at most 16 families some examples only
% \SetMathAlphabet{\mathrm}{normal}{OT1}{ppl}{m}{n}
% % \SetMathAlphabet{\mathrm}{bold}{OT1}{txr}{bx}{n}
% % \SetMathAlphabet{\mathrm}{varnormal}{OT1}{ppl}{m}{n}
% % \SetMathAlphabet{\mathrm}{varbold}{OT1}{ppl}{b}{n}

% \SetMathAlphabet{\mathit}{normal}{OT1}{ppl}{m}{it}
% % \SetMathAlphabet{\mathit}{bold}{OT1}{txr}{bx}{it}
% % \SetMathAlphabet{\mathit}{varnormal}{OT1}{ppl}{m}{it}
% % \SetMathAlphabet{\mathit}{varbold}{OT1}{ppl}{b}{it}

% \SetMathAlphabet{\mathbf}{normal}{\encodingdefault}{ppl}{bx}{n}%

% \renewcommand{\note}[2][1]{\mode<handout>{\oldnote{%

%       \hrulefill

%       \mdmath
%       \oldstylenums
%       \unsansmath
%       % \usefont{T1}{ppl}{m}{n} % linien figure (same height)
%       \usefont{T1}{pplj}{m}{n} % olf style figure

%       \textbf{Notes}

%       % \vspace*{1em}

%    \ifnum#1=0%
%       \twocolumn
%       #2 \vfill
%       \onecolumn
%    \else
%        \vspace*{1.1em}
%       \ifnum#1=1%
%          #2
%       \else
%          \begin{multicols}{2}
%          #2
%          \end{multicols}
%       \fi
%    \fi

% }}}

%%%%%%%%%%%%%%%%%%%%%%%%%%%%%%%%%%%%%%%%%%%%%%%%%%%%%%%%%%%%%%%%%%%%%%
